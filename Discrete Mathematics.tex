\documentclass{article}
\usepackage{inputenc}
\usepackage{graphicx, float}
\usepackage{amsmath, amssymb, amsthm}
\usepackage{enumitem}


\title{
    Discrete Mathematics \\
    \large by László Lovász and Katalin Vesztergombi \\ 
    Solutions to exercises}
\author{Francisco Milanich}
\date{}

\begin{document}

\maketitle
\newpage

\section{Introduction}

\section{Let us count!}

\subsection{Exercise}

How many ways can these people be seated at the table, if Alice too can sit any-
where?

\begin{align*}
    7 \cdot 6 \cdot 5 \cdot 4 \cdot 3 \cdot 2 \cdot 1 = 540
\end{align*}


\subsection{Exercise}

What is the number of “matchings” in Carl’s sense (when it matters who sits on
which side of the board, but the boards are all alike), and in Diane’s sense (when it is
the other way around)?

\begin{align*}
    &720 / 8 = 90 \\
    &720 / 6 = 120
\end{align*}

\subsection{Exercise}

Name sets whose elements are (a) buildings, (b) people, (c) students, (d) trees, (e)
numbers, (f) points.

\begin{enumerate}[label=\alph*\big)]
    \item B = \{Scarlet Devil Mansion, Hakurei Shrine\}
    \item P = \{Rumia, Cirno, Hong, Patchouli, Sakuya, Remilia, Flandre\}
    \item S = \{Reimu, Marisa\}
    \item T = \{Oak, Elm, Birch\}
    \item N = \{39, 35, 1024, 42, 151\}
    \item D = \{(2, 3, 1), (1, 4, 2), (0, 0, -1)\}
\end{enumerate}

\subsection{Exercise}

What are the elements of the following sets: (a) army, (b) mankind, (c) library, (d)
the animal kingdom?

\begin{enumerate}[label=\alph*\big)]
    \item A = \{Privates, Sergeants, Officers, Lieutenants, ...\}
    \item M = \{Tom Cruise, Lionel Messi, Linus Torvalds, Naruhito, ...\}
    \item L = \{Receptionist, Shelves, Chairs, Books, ...\}
    \item K = \{Tiger, Turtle, Fox, Raven, ...\}
\end{enumerate}

\subsection{Exercise}

Name sets having cardinality (a) 52, (b) 13, (c) 32, (d) 100, (e) 90, (f) 2,000,000.

\begin{enumerate}[label=\alph*\big)]
    \item D = \{Card deck\}
    \item S = \{Card suite\}
    \item P = \{Chess pieces\}
    \item Y = \{Years between 50 CE and AD 50\}
    \item M = \{Minutes of a football match\}
    \item D = \{Pixels in a 1250 $\times$ 1600 image\}
\end{enumerate}

\subsection{Exercise}

What are the elements of the following (admittedly peculiar) set: \{Alice, \{1\}\}?

\begin{enumerate}[label=\alph*\big)]
    \item Alice
    \item \{1\}
\end{enumerate}

\subsection{Exercise}

We have not written up all subset relations between various sets of numbers; for
example, $\mathbb{Z} \subseteq \mathbb{R}$ is also true. How many such relations can you find between the sets 
$\emptyset$, $\mathbb{N}$, $\mathbb{Z_+}$, $\mathbb{Z}$, $\mathbb{Q}$, $\mathbb{R}$?

\begin{enumerate}[label=\alph*\big)]
    \item $\mathbb{Z_+} \subseteq \mathbb{R}$
    \item $\mathbb{N} \subseteq \mathbb{R}$
    \item $\mathbb{\emptyset} \subseteq \mathbb{R}$
    \item $\mathbb{Z_+} \subseteq \mathbb{Q}$
    \item $\mathbb{N} \subseteq \mathbb{Q}$
    \item $\mathbb{\emptyset} \subseteq \mathbb{Q}$
    \item $\mathbb{N} \subseteq \mathbb{Z}$
    \item $\mathbb{\emptyset} \subseteq \mathbb{Z}$
    \item $\mathbb{\emptyset} \subseteq \mathbb{Z_+}$
\end{enumerate}

\end{document}

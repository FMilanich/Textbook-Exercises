\documentclass{article}
\usepackage{inputenc}
\usepackage{graphicx, float, forest}
\usepackage{amsmath, amssymb, amsthm}
\usepackage{enumitem}

% Esto es para centrar cosas entre dos ecuaciones, muy útil
\newcommand*\centermathcell[1]{\omit\hfil$\displaystyle#1$\hfil\ignorespaces}


\title{
    Discrete Mathematics \\
    \large by László Lovász and Katalin Vesztergombi \\ 
    Solutions to exercises}
\author{Francisco Milanich}
\date{}

\begin{document}

\maketitle
\newpage

\section{Introduction}

\section{Let us count!}

\subsection{Exercise}

How many ways can these people be seated at the table, if Alice too can sit any-
where?

\hrulefill\vspace{2mm}

\begin{align*}
    7 \cdot 6 \cdot 5 \cdot 4 \cdot 3 \cdot 2 \cdot 1 &= 540
\end{align*}


\subsection{Exercise}

What is the number of “matchings” in Carl’s sense (when it matters who sits on
which side of the board, but the boards are all alike), and in Diane’s sense (when it is
the other way around)?

\hrulefill\vspace{2mm}

\begin{align*}
    720 / 6 &= 120 \\
    720 / 8 &= 90
\end{align*}

\subsection{Exercise}

Name sets whose elements are (a) buildings, (b) people, (c) students, (d) trees, (e)
numbers, (f) points.

\hrulefill\vspace{2mm}

\begin{enumerate}[label=\big(\alph*\big)]
    \item B = \{Scarlet Devil Mansion, Hakurei Shrine\}
    \item P = \{Rumia, Cirno, Hong, Patchouli, Sakuya, Remilia, Flandre\}
    \item S = \{Reimu, Marisa\}
    \item T = \{Oak, Elm, Birch\}
    \item $N = \{39, 35, 1024, 42, 151\}$
    \item $D = \{(2, 3, 1), (1, 4, 2), (0, 0, -1)\}$
\end{enumerate}

\subsection{Exercise}

What are the elements of the following sets: (a) army, (b) mankind, (c) library, (d)
the animal kingdom?

\hrulefill\vspace{2mm}

\begin{enumerate}[label=\big(\alph*\big)]
    \item A = \{Privates, Sergeants, Officers, Lieutenants, ...\}
    \item M = \{Tom Cruise, Lionel Messi, Linus Torvalds, Naruhito, ...\}
    \item L = \{Receptionist, Shelves, Chairs, Books, ...\}
    \item K = \{Tiger, Turtle, Fox, Raven, ...\}
\end{enumerate}

\subsection{Exercise}

Name sets having cardinality (a) 52, (b) 13, (c) 32, (d) 100, (e) 90, (f) 2,000,000.

\hrulefill\vspace{2mm}

\begin{enumerate}[label=\big(\alph*\big)]
    \item D = \{Card deck\}
    \item S = \{Card suite\}
    \item P = \{Chess pieces\}
    \item Y = \{Years between 50 CE and AD 50\}
    \item M = \{Minutes of a football match\}
    \item D = \{Pixels in a 1250 $\times$ 1600 image\}
\end{enumerate}

\subsection{Exercise}

What are the elements of the following (admittedly peculiar) set: \{Alice, $\{1\}$\}?

\hrulefill\vspace{2mm}

\begin{enumerate}[label=\big(\alph*\big)]
    \item Alice
    \item $\{1\}$
\end{enumerate}

\subsection{Exercise}

We have not written up all subset relations between various sets of numbers; for
example, $\mathbb{Z} \subseteq \mathbb{R}$ is also true. How many such relations can you find between the sets 
$\emptyset$, $\mathbb{N}$, $\mathbb{Z_+}$, $\mathbb{Z}$, $\mathbb{Q}$, $\mathbb{R}$?

\hrulefill\vspace{2mm}

\begin{enumerate}[label=\big(\alph*\big)]
    \item $\mathbb{Z_+} \subseteq \mathbb{R}$
    \item $\mathbb{N} \subseteq \mathbb{R}$
    \item $\mathbb{\emptyset} \subseteq \mathbb{R}$
    \item $\mathbb{Z_+} \subseteq \mathbb{Q}$
    \item $\mathbb{N} \subseteq \mathbb{Q}$
    \item $\mathbb{\emptyset} \subseteq \mathbb{Q}$
    \item $\mathbb{N} \subseteq \mathbb{Z}$
    \item $\mathbb{\emptyset} \subseteq \mathbb{Z}$
    \item $\mathbb{\emptyset} \subseteq \mathbb{Z_+}$
\end{enumerate}

\subsection{Exercise}

Is an ``element of a set'' a special case of a ``subset of a set''?

\hrulefill\vspace{2mm}

No. Elements are not necessarily sets, and subsets are just sets with a specific property.
An ``element of a set'' can be a ``subset of a set'', but it isn't true in all cases.

\subsection{Exercise}

List all subsets of $\{0, 1, 3\}$. How many do you get?

\hrulefill\vspace{2mm}

\begin{enumerate}[label=\big(\alph*\big)]
    \item $\{0, 1, 3\}$
    \item $\{0, 1\}$
    \item $\{0, 3\}$
    \item $\{1, 3\}$
    \item $\{0\}$
    \item $\{1\}$
    \item $\{3\}$
    \item $\emptyset$
\end{enumerate}

\subsection{Exercise}

Define at least three sets, of which \{Alice, Diane, Eve\} is a subset.

\hrulefill\vspace{2mm}

\begin{enumerate}[label=\big(\alph*\big)]
    \item \{Alice, Diane, Eve\}
    \item \{Alice, Bob, Carl, Diane, Eve, Frank, George\}
    \item \{Alice, Diane, Eve, Francisco\}
\end{enumerate}

\subsection{Exercise}

List all subsets of $\{a, b, c, d, e\}$, containing $a$ but not containing $b$.

\hrulefill\vspace{2mm}

\begin{enumerate}[label=\big(\alph*\big)]
    \item $\{a, c, d, e\}$
    \item $\{a, c, d\}$
    \item $\{a, c, e\}$
    \item $\{a, d, e\}$
    \item $\{a, c\}$
    \item $\{a, d\}$
    \item $\{a, e\}$
    \item $\{a\}$
\end{enumerate}

\subsection{Exercise}

Define a set, of which both $\{1, 3, 4\}$ and $\{0, 3, 5\}$ are subsets. Find such a set with
a smallest possible number of elements.

\hrulefill\vspace{2mm}

\begin{align*}
    \{0, 1, 3, 4, 5\}
\end{align*}

\subsection{Exercise}

\begin{enumerate}[label=\big(\alph*\big)]
    \item Which set would you call the union of $\{a, b, c\}$, $\{a, b, d\}$ and $\{b, c, d, e\}$?
    \item Find the union of the first two sets, and then the union of this with the third. Also, find the union of the last two sets, and then the union of this with the first set. Try to formulate what you observed.
    \item Give a definition of the union of more than two sets.
\end{enumerate}

\hrulefill\vspace{2mm}

\begin{enumerate}[label=\big(\alph*\big)]
    \item $\{a, b, c, d, e\}$
    \item 
    \begin{align*}
        \{a, b, c\} \cup \{a, b, d\} &= \{a, b, c, d\} \\
        \{a, b, c, d\} \cup \{b, c, d, e\} &= \{a, b, c, d, e\} \\
        \\
        \{a, b, d\} \cup \{b, c, d, e\} &= \{a, b, c, d, e\} \\
        \{a, b, c, d, e\} \cup \{a, b, c\} &= \{a, b, c, d, e\} \\
    \end{align*}

    I observe that it's not important the order in which you do it, the order of the factors doesn't change the result.
    Also, the union of a set and a subset of such set results in the same set.
    \item The union of more than two sets is the set which contains all the elements in every set of the operation.
\end{enumerate}

\subsection{Exercise}

Explain the connection beween the notion of the union of sets and exercise 2.2.

\hrulefill\vspace{2mm}

The connection is that the union of sets and the operation for the number of ``matchings'' are combinations of elements.
That is, in both cases we are grouping and not counting individual things.

\subsection{Exercise}

We form the union of a set with 5 elements and a set with 9 elements. Which of
the following numbers can we get as the cardinality of the union: 4, 6, 9, 10, 14, 20?

\hrulefill\vspace{2mm}

We can get a set with cardinality 9, 10 or 14. The number depend on the number of duplicate elements in both sets.

\subsection{Exercise}

We form the union of two sets. We know that one of them has $n$ elements and
the other has $m$ elements. What can we infer for the cardinality of the union?

\hrulefill\vspace{2mm}

We can infer that the cardinality of the union has to be greater than $n$ and $m$,
and that it can be no greater than $n+m$.

\subsection{Exercise}

What is the intersection of
\begin{enumerate}[label=\big(\alph*\big)]
    \item The sets $\{0, 1, 3\}$ and $\{1, 2, 3\}$;
    \item the set of girls in this class and the set of boys in this class;
    \item the set of prime numbers and the set of even numbers?
\end{enumerate}

\hrulefill\vspace{2mm}

\begin{enumerate}[label=\big(\alph*\big)]
    \item $\{0, 1, 3\} \cap \{1, 2, 3\} = \{1, 3\}$
    \item $\{\text{Reimu}, \text{Marisa}\} \cap \emptyset = \emptyset$
    \item $\{\text{Prime numbers}\} \cap \{\text{Even numbers}\} = \{2\}$
\end{enumerate}

\subsection{Exercise}

We form the intersection of two sets. We know that one of them has $n$ elements
and the other has $m$ elements. What can we infer for the cardinality of the intersection?

\hrulefill\vspace{2mm}

We can infer that the cardinality of the intersection can be even $\emptyset$,
and that it can be no greater than the smallest number between $n$ and $m$.

\subsection{Exercise}

Prove that $|A \cup B| + |A \cap B| = |A| + |B|$.

\hrulefill\vspace{2mm}

\begin{align*}
    |A| &= |{\text{Elements only in }A}| + |A \cap B| \\
    |B| &= |{\text{Elements only in }B}| + |A \cap B| \\
    |A| + |B| &= |{\text{Elements only in }A}| + |{\text{Elements only in }B}| + 2|A \cap B| \\
    |A| + |B| &= |A \cup B| + |A \cap B|
\end{align*}

\subsection{Exercise}

The \textit{symmetric difference} of two sets $A$ and $B$ is the set of elements that belong
to exectly one of $A$ and $B$.
\begin{enumerate}[label=\big(\alph*\big)]
    \item What is the symmetric difference of the set $\mathbb{Z_+}$ of non-negative integers and the set $E$ of even integers ($E = \{...-4, -2, 0, 2, 4,...$ contains both negative and positive even integers$\}$).
    \item Form the symmetric difference of $A$ ad $B$, to get the set $C$. Form the symmetric difference of $A$ and $C$. What did you get? Give a proof of the answer.
\end{enumerate}

\hrulefill\vspace{2mm}

\begin{enumerate}[label=\big(\alph*\big)]
    \item The simmetric difference in that case would be the set of negative even numbers including 0 ($E^0_-$)
    \item From the symmetric difference of $A$ and $C$ I get the set $B$. That's because $C$ is the symmetric difference between $A$ and $B$, and because of that every element of $A$ appears twice in the successive symmetric differences and is therefore canceled out, while exactly the elements of $B$ remain.
\end{enumerate}

\subsection{Exercise}

Under the correspondence between numbers and subsets described above, which
number correspond to subsets with 1 element?

\hrulefill\vspace{2mm}

The number who corresponds to subsets with 1 element is $2^1$ (or 2), because
every set with 1 element has two subsets: $\emptyset$ and itself.

\subsection{Exercise}

What is the number of subsets of a set with $n$ elements, containing a given element?

\hrulefill\vspace{2mm}

The number of subsets containg a specific element of a set is just the number of
combinations of all the other elements. The number of subsets is the same because they are
almost identical, just adding that element. So that number should be $2^{n-1}$.

\subsection{Exercise}

What is the number of integers with
\begin{enumerate}[label=\big(\alph*\big)]
    \item at most $n$ (decimal) digits;
    \item exactly $n$ digits?
\end{enumerate}

\hrulefill\vspace{2mm}

\begin{enumerate}[label=\big(\alph*\big)]
    \item Because we are at base 10, the number is $10^n$. If we are counting negative numbers, the it would be $2 \cdot 10^n - 1$.
    \item Well, if we're counting $n=1$ as the numbers between 0 and 9, then it would be 10. But, since leading zeroes are not usually what we're looking for, there would be 9 options for every leading number and 10 for the rest
    \[
    \begin{cases}
        10 & n = 1 \\
        9 \cdot 10^{n-1} & n \geq 2
    \end{cases}
    \]
    And, if we count negative numbers it would be $2 \cdot 10 - 1 = 19$ for the numbers from -9 to 9, and $2 \cdot 9 \cdot 10^{n-1} = 18 \cdot 10^{n-1}$ for the rest
    \[
    \begin{cases}
        19 & n = 1 \\
        18 \cdot 10^{n-1} & n \geq 2
    \end{cases}
    \]
\end{enumerate}

\subsection{Exercise}

How many bits (binary digits) does $2^{100}$ have if written in base 2?

\hrulefill\vspace{2mm}

It'd have a 1 and 100 trailing zeroes. Thus, $2^{100}$ has $101$ bits.

\subsection{Exercise}

Find a formula for the number of digits of $2^n$.

\hrulefill\vspace{2mm}

Given a specific exponent of 2, if it has $k$ digits then it means that it's between $10^{k-1}$ and $10^k$ such that

\begin{align*}
    10^{k-1} \leqslant  2^n < 10^k
\end{align*}

Now if we rewrite $2^n$ as $10^x$, then $x = \log2^n = n\log2$ and follows that

\begin{alignat*}{4}
    \centermathcell{10^{k-1}} &\leqslant& \centermathcell{10^{n\log2}} &<& \centermathcell{10^k} \\
    \centermathcell{k-1} &\leqslant& \centermathcell{n\log2} &<& \centermathcell{k}
\end{alignat*}

And thus, knowing that $k - 1$ is an integer and $n\log2$ isn't, we get that $\lfloor n\log2 \rfloor + 1$ is the number of digits of $2^n$. Also, because $n\log2$ would never be an integer and $2^n \neq 10^k$, it's also true that the number of digits of $2^n$ is $\lceil n\log2 \rceil$.

\subsection{Exercise}

Draw a tree illustrating the way we counted the number of strings of length 2
formed from the characters $a$, $b$ and $c$, and explain how it gives the answer. Do the
same for the more general problem when $n=3$, $k_1=2$, $k_2=3$, $k_3=2$.

\hrulefill\vspace{2mm}

\begin{enumerate}[label=\big(\alph*\big)]
    \item  
        \begin{forest}
            [$a$ [$a$ [$aa$]] [$b$ [$ab$]] [$c$ [$ac$]]]
        \end{forest}
        \begin{forest}
            [$b$ [$a$ [$ba$]] [$b$ [$bb$]] [$c$ [$bc$]]]        
        \end{forest}
        \begin{forest}
            [$c$ [$a$ [$ca$]] [$b$ [$cb$]] [$c$ [$cc$]]]        
        \end{forest}
    \item $k_1$ = $\{a_1, b_1\}$ \\ $k_2$ = $\{a_2, b_2, c_2\}$ \\ $k_3$ = $\{a_3, b_3\}$ \\
        \begin{forest}
            [$a_1$ [$a_2$ [$a_3$ [$a_1a_2a_3$]] [$b_3$ [$a_1a_2b_3$]]] [$b_2$ [$a_3$ [$a_1b_2a_3$]] [$b_3$ [$a_1b_2b_3$]]] [$c_2$ [$a_3$ [$a_1c_2a_3$]] [$b_3$ [$a_1c_2b_3$]]]]
        \end{forest} \\
        \begin{forest}
            [$b_1$ [$a_2$ [$a_3$ [$b_1a_2a_3$]] [$b_3$ [$b_1a_2b_3$]]] [$b_2$ [$a_3$ [$b_1b_2a_3$]] [$b_3$ [$b_1b_2b_3$]]] [$c_2$ [$a_3$ [$b_1c_2a_3$]] [$b_3$ [$b_1c_2b_3$]]]]
        \end{forest}
\end{enumerate}

\subsection{Exercise}

In a sport shop, there are T-shirts of 5 different colors, shorts of 4 different colors,
and socks of 3 different colors. How many different uniforms can you compose from
these items?

\hrulefill\vspace{2mm}

\begin{align*}
    5 \cdot 4 \cdot 3 = 60
\end{align*}

\subsection{Exercise}

On a ticket for a succer sweepstake, you have to guess 1, 2, or X for each of 13
games. How many different ways can you fill out the ticket?

\hrulefill\vspace{2mm}

\begin{align*}
    3^{13} = 1594323
\end{align*}

\end{document}

\documentclass{article}
\usepackage{inputenc, caption}
\usepackage{graphicx, float, forest}
\usepackage{amsmath, amssymb, amsthm}
\usepackage{enumitem}

% Esto es para centrar cosas entre dos ecuaciones, muy útil
\newcommand*\centermathcell[1]{\omit\hfil$\displaystyle#1$\hfil\ignorespaces}


\title{
    Discrete Mathematics \\
    \large by László Lovász and Katalin Vesztergombi \\ 
    Solutions to exercises}
\author{Francisco Milanich}
\date{}

\begin{document}

\maketitle
\newpage

\section{Introduction}

\section{Let us count!}

\subsection{Exercise}

How many ways can these people be seated at the table, if Alice too can sit any-
where?

\hrulefill\vspace{2mm}

\begin{align*}
    7 \cdot 6 \cdot 5 \cdot 4 \cdot 3 \cdot 2 \cdot 1 &= 540
\end{align*}


\subsection{Exercise}

What is the number of “matchings” in Carl’s sense (when it matters who sits on
which side of the board, but the boards are all alike), and in Diane’s sense (when it is
the other way around)?

\hrulefill\vspace{2mm}

\begin{align*}
    720 / 6 &= 120 \\
    720 / 8 &= 90
\end{align*}

\subsection{Exercise}

Name sets whose elements are (a) buildings, (b) people, (c) students, (d) trees, (e)
numbers, (f) points.

\hrulefill\vspace{2mm}

\begin{enumerate}[label=\big(\alph*\big)]
    \item B = \{Scarlet Devil Mansion, Hakurei Shrine\}
    \item P = \{Rumia, Cirno, Hong, Patchouli, Sakuya, Remilia, Flandre\}
    \item S = \{Reimu, Marisa\}
    \item T = \{Oak, Elm, Birch\}
    \item $N = \{39, 35, 1024, 42, 151\}$
    \item $D = \{(2, 3, 1), (1, 4, 2), (0, 0, -1)\}$
\end{enumerate}

\subsection{Exercise}

What are the elements of the following sets: (a) army, (b) mankind, (c) library, (d)
the animal kingdom?

\hrulefill\vspace{2mm}

\begin{enumerate}[label=\big(\alph*\big)]
    \item A = \{Privates, Sergeants, Officers, Lieutenants, ...\}
    \item M = \{Tom Cruise, Lionel Messi, Linus Torvalds, Naruhito, ...\}
    \item L = \{Receptionist, Shelves, Chairs, Books, ...\}
    \item K = \{Tiger, Turtle, Fox, Raven, ...\}
\end{enumerate}

\subsection{Exercise}

Name sets having cardinality (a) 52, (b) 13, (c) 32, (d) 100, (e) 90, (f) 2,000,000.

\hrulefill\vspace{2mm}

\begin{enumerate}[label=\big(\alph*\big)]
    \item D = \{Card deck\}
    \item S = \{Card suite\}
    \item P = \{Chess pieces\}
    \item Y = \{Years between 50 CE and AD 50\}
    \item M = \{Minutes of a football match\}
    \item D = \{Pixels in a 1250 $\times$ 1600 image\}
\end{enumerate}

\subsection{Exercise}

What are the elements of the following (admittedly peculiar) set: \{Alice, $\{1\}$\}?

\hrulefill\vspace{2mm}

\begin{enumerate}[label=\big(\alph*\big)]
    \item Alice
    \item $\{1\}$
\end{enumerate}

\subsection{Exercise}

We have not written up all subset relations between various sets of numbers; for
example, $\mathbb{Z} \subseteq \mathbb{R}$ is also true. How many such relations can you find between the sets 
$\emptyset$, $\mathbb{N}$, $\mathbb{Z_+}$, $\mathbb{Z}$, $\mathbb{Q}$, $\mathbb{R}$?

\hrulefill\vspace{2mm}

\begin{enumerate}[label=\big(\alph*\big)]
    \item $\mathbb{Z_+} \subseteq \mathbb{R}$
    \item $\mathbb{N} \subseteq \mathbb{R}$
    \item $\mathbb{\emptyset} \subseteq \mathbb{R}$
    \item $\mathbb{Z_+} \subseteq \mathbb{Q}$
    \item $\mathbb{N} \subseteq \mathbb{Q}$
    \item $\mathbb{\emptyset} \subseteq \mathbb{Q}$
    \item $\mathbb{N} \subseteq \mathbb{Z}$
    \item $\mathbb{\emptyset} \subseteq \mathbb{Z}$
    \item $\mathbb{\emptyset} \subseteq \mathbb{Z_+}$
\end{enumerate}

\subsection{Exercise}

Is an ``element of a set'' a special case of a ``subset of a set''?

\hrulefill\vspace{2mm}

No. Elements are not necessarily sets, and subsets are just sets with a specific property.
An ``element of a set'' can be a ``subset of a set'', but it isn't true in all cases.

\subsection{Exercise}

List all subsets of $\{0, 1, 3\}$. How many do you get?

\hrulefill\vspace{2mm}

\begin{enumerate}[label=\big(\alph*\big)]
    \item $\{0, 1, 3\}$
    \item $\{0, 1\}$
    \item $\{0, 3\}$
    \item $\{1, 3\}$
    \item $\{0\}$
    \item $\{1\}$
    \item $\{3\}$
    \item $\emptyset$
\end{enumerate}

\subsection{Exercise}

Define at least three sets, of which \{Alice, Diane, Eve\} is a subset.

\hrulefill\vspace{2mm}

\begin{enumerate}[label=\big(\alph*\big)]
    \item \{Alice, Diane, Eve\}
    \item \{Alice, Bob, Carl, Diane, Eve, Frank, George\}
    \item \{Alice, Diane, Eve, Francisco\}
\end{enumerate}

\subsection{Exercise}

List all subsets of $\{a, b, c, d, e\}$, containing $a$ but not containing $b$.

\hrulefill\vspace{2mm}

\begin{enumerate}[label=\big(\alph*\big)]
    \item $\{a, c, d, e\}$
    \item $\{a, c, d\}$
    \item $\{a, c, e\}$
    \item $\{a, d, e\}$
    \item $\{a, c\}$
    \item $\{a, d\}$
    \item $\{a, e\}$
    \item $\{a\}$
\end{enumerate}

\subsection{Exercise}

Define a set, of which both $\{1, 3, 4\}$ and $\{0, 3, 5\}$ are subsets. Find such a set with
a smallest possible number of elements.

\hrulefill\vspace{2mm}

\begin{align*}
    \{0, 1, 3, 4, 5\}
\end{align*}

\subsection{Exercise}

\begin{enumerate}[label=\big(\alph*\big)]
    \item Which set would you call the union of $\{a, b, c\}$, $\{a, b, d\}$ and $\{b, c, d, e\}$?
    \item Find the union of the first two sets, and then the union of this with the third. Also, find the union of the last two sets, and then the union of this with the first set. Try to formulate what you observed.
    \item Give a definition of the union of more than two sets.
\end{enumerate}

\hrulefill\vspace{2mm}

\begin{enumerate}[label=\big(\alph*\big)]
    \item $\{a, b, c, d, e\}$
    \item 
    \begin{align*}
        \{a, b, c\} \cup \{a, b, d\} &= \{a, b, c, d\} \\
        \{a, b, c, d\} \cup \{b, c, d, e\} &= \{a, b, c, d, e\} \\
        \\
        \{a, b, d\} \cup \{b, c, d, e\} &= \{a, b, c, d, e\} \\
        \{a, b, c, d, e\} \cup \{a, b, c\} &= \{a, b, c, d, e\} \\
    \end{align*}

    I observe that it's not important the order in which you do it, the order of the factors doesn't change the result.
    Also, the union of a set and a subset of such set results in the same set.
    \item The union of more than two sets is the set which contains all the elements in every set of the operation.
\end{enumerate}

\subsection{Exercise}

Explain the connection beween the notion of the union of sets and exercise 2.2.

\hrulefill\vspace{2mm}

The connection is that the union of sets and the operation for the number of ``matchings'' are combinations of elements.
That is, in both cases we are grouping and not counting individual things.

\subsection{Exercise}

We form the union of a set with 5 elements and a set with 9 elements. Which of
the following numbers can we get as the cardinality of the union: 4, 6, 9, 10, 14, 20?

\hrulefill\vspace{2mm}

We can get a set with cardinality 9, 10 or 14. The number depend on the number of duplicate elements in both sets.

\subsection{Exercise}

We form the union of two sets. We know that one of them has $n$ elements and
the other has $m$ elements. What can we infer for the cardinality of the union?

\hrulefill\vspace{2mm}

We can infer that the cardinality of the union has to be greater than $n$ and $m$,
and that it can be no greater than $n+m$.

\subsection{Exercise}

What is the intersection of
\begin{enumerate}[label=\big(\alph*\big)]
    \item The sets $\{0, 1, 3\}$ and $\{1, 2, 3\}$;
    \item the set of girls in this class and the set of boys in this class;
    \item the set of prime numbers and the set of even numbers?
\end{enumerate}

\hrulefill\vspace{2mm}

\begin{enumerate}[label=\big(\alph*\big)]
    \item $\{0, 1, 3\} \cap \{1, 2, 3\} = \{1, 3\}$
    \item $\{\text{Reimu}, \text{Marisa}\} \cap \emptyset = \emptyset$
    \item $\{\text{Prime numbers}\} \cap \{\text{Even numbers}\} = \{2\}$
\end{enumerate}

\subsection{Exercise}

We form the intersection of two sets. We know that one of them has $n$ elements
and the other has $m$ elements. What can we infer for the cardinality of the intersection?

\hrulefill\vspace{2mm}

We can infer that the cardinality of the intersection can be even $\emptyset$,
and that it can be no greater than the smallest number between $n$ and $m$.

\subsection{Exercise}

Prove that $|A \cup B| + |A \cap B| = |A| + |B|$.

\hrulefill\vspace{2mm}

\begin{align*}
    |A| &= |{\text{Elements only in }A}| + |A \cap B| \\
    |B| &= |{\text{Elements only in }B}| + |A \cap B| \\
    |A| + |B| &= |{\text{Elements only in }A}| + |{\text{Elements only in }B}| + 2|A \cap B| \\
    |A| + |B| &= |A \cup B| + |A \cap B|
\end{align*}

\subsection{Exercise}

The \textit{symmetric difference} of two sets $A$ and $B$ is the set of elements that belong
to exectly one of $A$ and $B$.
\begin{enumerate}[label=\big(\alph*\big)]
    \item What is the symmetric difference of the set $\mathbb{Z_+}$ of non-negative integers and the set $E$ of even integers ($E = \{...-4, -2, 0, 2, 4,...$ contains both negative and positive even integers$\}$).
    \item Form the symmetric difference of $A$ ad $B$, to get the set $C$. Form the symmetric difference of $A$ and $C$. What did you get? Give a proof of the answer.
\end{enumerate}

\hrulefill\vspace{2mm}

\begin{enumerate}[label=\big(\alph*\big)]
    \item The simmetric difference in that case would be the set of negative even numbers including 0 ($E^0_-$)
    \item From the symmetric difference of $A$ and $C$ I get the set $B$. That's because $C$ is the symmetric difference between $A$ and $B$, and because of that every element of $A$ appears twice in the successive symmetric differences and is therefore canceled out, while exactly the elements of $B$ remain.
\end{enumerate}

\subsection{Exercise}

Under the correspondence between numbers and subsets described above, which
number correspond to subsets with 1 element?

\hrulefill\vspace{2mm}

The number who corresponds to subsets with 1 element is $2^1$ (or 2), because
every set with 1 element has two subsets: $\emptyset$ and itself.

\subsection{Exercise}

What is the number of subsets of a set with $n$ elements, containing a given element?

\hrulefill\vspace{2mm}

The number of subsets containg a specific element of a set is just the number of
combinations of all the other elements. The number of subsets is the same because they are
almost identical, just adding that element. So that number should be $2^{n-1}$.

\subsection{Exercise}

What is the number of integers with
\begin{enumerate}[label=\big(\alph*\big)]
    \item at most $n$ (decimal) digits;
    \item exactly $n$ digits?
\end{enumerate}

\hrulefill\vspace{2mm}

\begin{enumerate}[label=\big(\alph*\big)]
    \item Because we are at base 10, the number is $10^n$. If we are counting negative numbers, the it would be $2 \cdot 10^n - 1$.
    \item Well, if we're counting $n=1$ as the numbers between 0 and 9, then it would be 10. But, since leading zeroes are not usually what we're looking for, there would be 9 options for every leading number and 10 for the rest
    \[
    \begin{cases}
        10 & n = 1 \\
        9 \cdot 10^{n-1} & n \geq 2
    \end{cases}
    \]
    And, if we count negative numbers it would be $2 \cdot 10 - 1 = 19$ for the numbers from -9 to 9, and $2 \cdot 9 \cdot 10^{n-1} = 18 \cdot 10^{n-1}$ for the rest
    \[
    \begin{cases}
        19 & n = 1 \\
        18 \cdot 10^{n-1} & n \geq 2
    \end{cases}
    \]
\end{enumerate}

\subsection{Exercise}

How many bits (binary digits) does $2^{100}$ have if written in base 2?

\hrulefill\vspace{2mm}

It'd have a 1 and 100 trailing zeroes. Thus, $2^{100}$ has $101$ bits.

\subsection{Exercise}

Find a formula for the number of digits of $2^n$.

\hrulefill\vspace{2mm}

Given a specific exponent of 2, if it has $k$ digits then it means that it's between $10^{k-1}$ and $10^k$ such that

\begin{align*}
    10^{k-1} \leqslant  2^n < 10^k
\end{align*}

Now if we rewrite $2^n$ as $10^x$, then $x = \log2^n = n\log2$ and follows that

\begin{alignat*}{4}
    \centermathcell{10^{k-1}} &\leqslant& \centermathcell{10^{n\log2}} &<& \centermathcell{10^k} \\
    \centermathcell{k-1} &\leqslant& \centermathcell{n\log2} &<& \centermathcell{k}
\end{alignat*}

And thus, knowing that $k - 1$ is an integer and $n\log2$ isn't, we get that $\lfloor n\log2 \rfloor + 1$ is the number of digits of $2^n$. Also, because $n\log2$ would never be an integer and $2^n \neq 10^k$, it's also true that the number of digits of $2^n$ is $\lceil n\log2 \rceil$.

\subsection{Exercise}

Draw a tree illustrating the way we counted the number of strings of length 2
formed from the characters $a$, $b$ and $c$, and explain how it gives the answer. Do the
same for the more general problem when $n=3$, $k_1=2$, $k_2=3$, $k_3=2$.

\hrulefill\vspace{2mm}

\begin{enumerate}[label=\big(\alph*\big)]
    \item  
        \begin{forest}
            [$a$ [$a$ [$aa$]] [$b$ [$ab$]] [$c$ [$ac$]]]
        \end{forest}
        \begin{forest}
            [$b$ [$a$ [$ba$]] [$b$ [$bb$]] [$c$ [$bc$]]]        
        \end{forest}
        \begin{forest}
            [$c$ [$a$ [$ca$]] [$b$ [$cb$]] [$c$ [$cc$]]]        
        \end{forest}
    \item $k_1$ = $\{a_1, b_1\}$ \\ $k_2$ = $\{a_2, b_2, c_2\}$ \\ $k_3$ = $\{a_3, b_3\}$ \\
        \begin{forest}
            [$a_1$ [$a_2$ [$a_3$ [$a_1a_2a_3$]] [$b_3$ [$a_1a_2b_3$]]] [$b_2$ [$a_3$ [$a_1b_2a_3$]] [$b_3$ [$a_1b_2b_3$]]] [$c_2$ [$a_3$ [$a_1c_2a_3$]] [$b_3$ [$a_1c_2b_3$]]]]
        \end{forest} \\
        \begin{forest}
            [$b_1$ [$a_2$ [$a_3$ [$b_1a_2a_3$]] [$b_3$ [$b_1a_2b_3$]]] [$b_2$ [$a_3$ [$b_1b_2a_3$]] [$b_3$ [$b_1b_2b_3$]]] [$c_2$ [$a_3$ [$b_1c_2a_3$]] [$b_3$ [$b_1c_2b_3$]]]]
        \end{forest}
\end{enumerate}

\subsection{Exercise}

In a sport shop, there are T-shirts of 5 different colors, shorts of 4 different colors,
and socks of 3 different colors. How many different uniforms can you compose from
these items?

\hrulefill\vspace{2mm}

\begin{align*}
    5 \cdot 4 \cdot 3 = 60
\end{align*}

\subsection{Exercise}

On a ticket for a succer sweepstake, you have to guess 1, 2, or X for each of 13
games. How many different ways can you fill out the ticket?

\hrulefill\vspace{2mm}

\begin{align*}
    3^{13} = 1594323
\end{align*}

\subsection{Exercise}

We roll a dice twice; how many different outcomes can we have (a 1 followed by
a 4 is different from a 4 followed by a 1)?

\hrulefill\vspace{2mm}

\begin{align*}
    6^{2} = 36
\end{align*}

\subsection{Exercise}

We have 20 different presents that we want to distribute to 12 children. It is
not required that every child gets something; it could even happen that we give all the
presents to the same child. In how many ways can we distribute the presents?

\hrulefill\vspace{2mm}

\begin{align*}
    12^{20} = 3833759992447475122176
\end{align*}

\subsection{Exercise}

We have 20 kinds of presents; this time, we have a large supply from each. We
want to give presents to 12 children. Again, it is not required that every child gets
something; but no child can get two copies of the same present. In how many ways can
we give presents?

\hrulefill\vspace{2mm}

Because a child can either receive a present or not, there's 2 options in a sequence of 20. 
But then, because there's 12 children, that same sequence has to be repeated 12 times giving us:

\begin{align*}
    (2^{20})^{12} = 2^{240} \approx 1.767 \times 10^{72}
\end{align*}

\subsection{Exercise}

$n$ boys and $n$ girls go out to dance. In how many ways can they all dance
simultaneously? (We assume that only couples of different sex dance with each other.)

\hrulefill\vspace{2mm}

Because there's $n$ boys for $n$ girls, the matchings are exactly one-to-one. The answer is $n!$.

\subsection{Exercise}

\begin{enumerate}[label=\big(\alph*\big)]
    \item Draw a tree for Alice’s solution of enumerating the number of ways 6 people can play chess, and explain Alice’s argument using the tree.
    \item Solve the problem for 8 people. Can you give a general formula for 2$n$ people?
\end{enumerate}

\hrulefill\vspace{2mm}

% At this point I regretted so much deciding to do the trees like this...

\begin{enumerate}[label=\big(\alph*\big)]
    \item I'm just going to do the first levels because it doesn't fit, but it illustrates the reasoning. Also, there would be too many nodes in the tree, as the result is $6! = 720$. \\ \\
        \begin{forest}
            [Bob [Carl [$\cdots$]] [Diane [$\cdots$]] [Eve [$\cdots$]] [Frank [$\cdots$]] [George [$\cdots$]]]
        \end{forest}
        \begin{forest}
            [Carl [Bob [$\cdots$]] [Diane [$\cdots$]] [Eve [$\cdots$]] [Frank [$\cdots$]] [George [$\cdots$]]]
        \end{forest}
        \begin{forest}
            [Diane [Bob [$\cdots$]] [Carl [$\cdots$]] [Eve [$\cdots$]] [Frank [$\cdots$]] [George [$\cdots$]]]
        \end{forest}
        \begin{forest}
            [Eve [Bob [$\cdots$]] [Carl [$\cdots$]] [Diane [$\cdots$]] [Frank [$\cdots$]] [George [$\cdots$]]]
        \end{forest}
        \begin{forest}
            [Frank [Bob [$\cdots$]] [Carl [$\cdots$]] [Diane [$\cdots$]] [Eve [$\cdots$]] [George [$\cdots$]]]
        \end{forest}
        \begin{forest}
            [George [Bob [$\cdots$]] [Carl [$\cdots$]] [Diane [$\cdots$]] [Eve [$\cdots$]] [Frank [$\cdots$]]]
        \end{forest}
    \item Because it's just a permutation, the result for 8 people is $8! = 40320$. For $2n$ people the general formula is just $(2n)!$.
\end{enumerate}

\subsection{Exercise}

\begin{enumerate}[label=\big(\alph*\big)]
    \item Which is larger, $n$ or $n(n-1)/2$?
    \item Which is larger, $n^2$ or $2^n$?
\end{enumerate}

\hrulefill\vspace{2mm}

\begin{enumerate}[label=\big(\alph*\big)]
    \item Let's set up an inequality to verify this.
    \begin{align*}
        \frac{n(n-1)}{2} > n \\
        n(n-1) > 2n \\
        n^2-n > 2n \\
        n(n-3) > 0
    \end{align*}
    This holds up for $n=3$, so the answer would be
    \[\begin{cases}
        \frac{n(n-1)}{2} > n & n > 3 \\
        \frac{n(n-1)}{2} = n & n = 0, 3 \\
        \frac{n(n-1)}{2} < n & 0 \leqslant n < 3
    \end{cases}\]
    \item This time I'm going to set up a table, for it seems easier
    \begin{center}
    \begin{tabular}{||c c c||} 
    \hline
    $n$ & $n^2$ & $2^n$ \\ [0.5ex] 
    \hline\hline
    0 & 0 & 1 \\ 
    \hline
    1 & 1 & 2 \\ 
    \hline
    2 & 4 & 4 \\
    \hline
    3 & 9 & 8 \\
    \hline
    4 & 16 & 16 \\
    \hline
    5 & 25 & 32 \\
    \hline
    6 & 36 & 64 \\
    \hline
    \end{tabular}
    \end{center}
    As it seems that the tendency is clear, the relations would be
    \[\begin{cases}
        n^2 > 2^n & n = 3 \\
        n^2 = 2^n & n = 2, 4 \\
        n^2 < 2^n & n = 0, 1 \lor n \geq 5
    \end{cases}\]
\end{enumerate}

\subsection{Exercise}

\begin{enumerate}[label=\big(\alph*\big)]
    \item Prove that $2^n > n^3$ is $n$ is large enough.
    \item Use (a) to prove that $2^n/n^2$ becomes arbitrarily large if $n$ is large enough.
\end{enumerate}

\hrulefill\vspace{2mm}

\begin{enumerate}[label=\big(\alph*\big)]
    \item This can be seen when we look at the logarithms
    \begin{align*}
        2^n &> n^3 \\
        \log{2^n} &> \log{n^3} \\
        n \log{2} &> 3 \log{n}
    \end{align*}
    As we can see, $2^n$ grows at a linear growth and $n^3$ grows at a logarithmic growth. As is trivial to prove that that linear growth is faster, we can conclude that $2^n > n^3$ as $n \to \infty$.
    \item We can combine the previous inequality with the one proposed like this
    \begin{align*}
        2^n &> n^3 \\
        \frac{2^n}{n^2} &> \frac{n^3}{n^2} \\
        \frac{2^n}{n^2} &> n
    \end{align*}
    Because $n \to \infty$, and $2^n / n^2$ is bigger is greater than $n$, we can conclude that $2^n / n^2 \to \infty$.
\end{enumerate}

\section{Induction}

\subsection{Exercise}

Prove, using induction but also without it, that $n(n-1)$ is an even number for
every non-negative integer $n$.

\hrulefill\vspace{2mm}

This is quite easy to prove without induction, as every integer is either even or odd. Also, we know that if an integer has an even factor, then it itself is an even number. So, if $n$ is even, then $n(n-1)$ is even, and if $n$ is odd, then $(n-1)$ and $n(n-1)$ are even. \\
But, if we are going to prove this by induction, then we need to be a bit more specific. We'll use as the induction hypothesis that $(n-1)(n-2)$ is an even integer and proceed like this

\begin{align*}
    n(n-1) &= [(n-2)+2](n-1) \\
    n(n-1) &= (n-1)(n-2)+2(n-1)
\end{align*}

As we see, $2(n-1)$ is clearly even and we get that $(n-1)(n-2)$ is out induction hypothesis. As there's at least one even factor, we get that $n(n-1)$ is even.

\subsection{Exercise}

Prove by induction that the sum of the first $n$ positive integers is $n(n+1)/2$.

\hrulefill\vspace{2mm}

Given as our induction hypothesis that

\begin{align*}
    1 + 2 + \cdots + (n-1) = \frac{(n-1)n}{2} 
\end{align*}

Now, we can finally do the induction.

\begin{align*}
    1 + 2 + \cdots + (n-1) + n &= (1 + 2 + \cdots + (n-1)) + n \\ 
    1 + 2 + \cdots + (n-1) + n &= \left(\frac{(n-1)n}{2}\right) + n \\
    1 + 2 + \cdots + (n-1) + n &= \frac{n(n+1)}{2}
\end{align*}

\subsection{Exercise}

Observe that the number $n(n+1)/2$ is the number of handshakes among $n+1$ people. Suppose that everyone counts only handshakes with people older than him/her
(pretty snobbish, isn’t it?). Who will count the largest number of handshakes? How
many people count 6 handshakes?

\hrulefill\vspace{2mm}

Well, the youngest person in the room will count $n$ handshakes (the largest number), the second $n-1$, the third $n-2$, etc. So, given that $n \geqslant 6$, the 6th oldest person counts exactly 6 handshakes.

\subsection{Exercise}

Give a proof of exercise 3.1, based on figure 3.

\hrulefill\vspace{2mm}

\begin{figure*}[h]
    \includegraphics[width=\linewidth]{Discrete-Mathematics-images/fig3.jpg}
    \captionsetup{labelformat=empty}
    \caption{Figure 3: The sum of the first $n$ integers}
    \label{fig:fig3}
\end{figure*}

In exercise 3.1 we were asked to prove that $n(n-1)$ is an even number for every non-negative integer $n$.
In the given figure, we can see that for any $n$ consecutive positive integers (in this case there's 5), the product of that number by the next positive integer is even.
The figure is very graphical, as we can see that it makes a square which is clearly even.
So, $n(n+1)$ is always even, because it is a product of two consecutive positive integers.

\subsection{Exercise}

Prove the follwing identity:

\begin{align*}
    1 \cdot 2 + 2 \cdot 3 + 3 \cdot 4 + \cdots + (n-1) \cdot n = \frac{(n-1) \cdot n \cdot (n+1)}{3}
\end{align*}

\hrulefill\vspace{2mm}

We're going to take as an induction hypothesis that

\begin{align*}
    1 \cdot 2 + 2 \cdot 3 + 3 \cdot 4 + \cdots + (n-2) \cdot (n-1) = \frac{(n-2) \cdot (n-1) \cdot n}{3}
\end{align*}

With this in mind, let's do the induction

\begin{align*}
    1 \cdot 2 + 2 \cdot 3 + 3 \cdot 4 + \cdots + (n-1) \cdot n &= [1 \cdot 2 + 2 \cdot 3 + 3 \cdot 4 + \cdots + (n-2) \cdot (n-1)] + (n-1) \cdot n \\
    1 \cdot 2 + 2 \cdot 3 + 3 \cdot 4 + \cdots + (n-1) \cdot n &= \left[\frac{(n-2) \cdot(n-1) \cdot n}{3}\right] + (n-1) \cdot n \\
    1 \cdot 2 + 2 \cdot 3 + 3 \cdot 4 + \cdots + (n-1) \cdot n &= (n-1) \cdot n \cdot \left(\frac{n-2}{3} + 1\right) \\
    1 \cdot 2 + 2 \cdot 3 + 3 \cdot 4 + \cdots + (n-1) \cdot n &= (n-1) \cdot n \cdot \left(\frac{n-2+3}{3}\right) \\
    1 \cdot 2 + 2 \cdot 3 + 3 \cdot 4 + \cdots + (n-1) \cdot n &= \frac{(n-1) \cdot n \cdot (n+1)}{3}
\end{align*}

\subsection{Exercise}

Use the method of the little Gauss to give a third proof of the formula in exercise 3.1

\hrulefill\vspace{2mm}

Again, in exercise 3.1 we were asked to prove that $n(n-1)$ is an even number for every non-negative integer $n$.
Our little Gauss postulated something similar to this

\begin{align*}
    S &= 1 + 2 + 3 + \cdots + n \\
    S &= (1+n) + [2+(n-1)] + [3+(n-2)] + \cdots + \left[\frac{n}{2} + \left[ \left(\frac{n}{2}\right) + 1\right] \right]
\end{align*}

Knowing this, we can make up two cases, when $n$ is even and when $n$ is odd. If $n$ is even the pairs are exactly $n/2$, so we can develop it in a straightfoward way.

\begin{align*}
    S &= \frac{n}{2} \cdot (n+1) \\
    2S &= n(n+1)
\end{align*}

It's trivial to prove that $2S$ is even, so $n(n+1)$ is also even. Now, if $n$ is odd, then there's exactly $(n-1)/2$ pairs with one unpaired term $(n+1)/2$.

\begin{align*}
    S &= \frac{n-1}{2} \cdot (n+1) + \frac{n+1}{2} \\
    S &= \frac{n+1}{2} \cdot \left((n-1)+1\right) \\
    S &= \frac{n}{2} \cdot (n+1) \\
    2S &= n(n+1)
\end{align*}

We get to the same result as with the other case, so we can conclude that $n(n+1)$ is even for every positive integer $n$.

\subsection{Exercise}

How would the little Gauss prove the formula for the sum of the first $n$ odd numbers (2)?

\hrulefill\vspace{2mm}

Well, I guess the little Gauss would deduce a formula for the $n$th element like this

\begin{align*}
    S = 1 + 3 + \cdots + (2n - 3) + (2n - 1)
\end{align*}

Then he would pair the first with the last term, the second with the second to last, etc.

\begin{align*}
    S = [1 + (2n - 1)] + [3 + (2n -3)] + \cdots
\end{align*}

If we look at it, we get that every pair has the value $2n$ and there's exactly $n/2$ of them.
Knowing this, we can derive the formula we want to prove.

\begin{align*}
    S &= \frac{n}{2} \cdot 2n \\
    S &= n^2
\end{align*}

\subsection{Exercise}

Prove that the sum of the first $n$ squares $(1+4+9+\cdots+n^2)$ is $n(n+1)(2n+1)/6$.

\hrulefill\vspace{2mm}

To do this, I'm going to use as the induction hypothesis the following equation:

\begin{align*}
    1 + 4 + 9 + \cdots + (n-1)^2 &= \frac{(n-1)n(2(n-1)+1)}{6} \\
    1 + 4 + 9 + \cdots + (n-1)^2 &= \frac{(n-1)n(2n-1)}{6}
\end{align*}

Having this in mind, we can expand our original sequence of sums to prove the postulated formula.

\begin{align*}
    1 + 4 + 9 + \cdots + n^2 &= [1 + 4 + 9 + \cdots + (n-1)^2] + n^2 \\
    1 + 4 + 9 + \cdots + n^2 &= \frac{(n-1)n(2n-1)}{6} + n^2 \\
    1 + 4 + 9 + \cdots + n^2 &= \frac{n(2n^2-3n+1)+6n^2}{6} \\
    1 + 4 + 9 + \cdots + n^2 &= \frac{2n^3-3n^2+n+6n^2}{6} \\
    1 + 4 + 9 + \cdots + n^2 &= \frac{2n^3+3n^2+n}{6} \\
    1 + 4 + 9 + \cdots + n^2 &= \frac{n(2n^2+3n+1)}{6} \\
    1 + 4 + 9 + \cdots + n^2 &= \frac{n(n+1)(2n+1)}{6}
\end{align*}

\subsection{Exercise}

Prove that the sum of the first $n$ powers of 2 (starting with $1=2^0$) is $2^n-1$.

\hrulefill\vspace{2mm}

To prove this, I'm going to use as the induction hypothesis that

\begin{align*}
    2^0 + 2^1 + 2^2 + \cdots + 2^{n-1} = 2^{n} - 1
\end{align*}

With this in mind, let's expand the sequence of sums postulated.

\begin{align*}
    2^0 + 2^1 + 2^2 + \cdots + 2^{n} &= [2^0 + 2^1 + 2^2 + \cdots + 2^{n-1}] + 2^{n} \\
    2^0 + 2^1 + 2^2 + \cdots + 2^{n} &= (2^{n} - 1) + 2^{n} \\
    2^0 + 2^1 + 2^2 + \cdots + 2^{n} &= 2 \cdot 2^n - 1 \\
    2^0 + 2^1 + 2^2 + \cdots + 2^{n} &= 2^{n+1} \\
\end{align*}

This induction was quite different from the other ones, as I used as the induction hypothesis the proposed formula, but as it acomplishes the case for $n+1$, then it's true for all cases.
I could have probably done this with $n-2$, but I felt this was a more natural way to do it.

\subsection{Exercise}

Use induction to prove Theorem 2.2 (the number of strings of length $n$ composed of $k$ given elements is $k^n$) and Theorem 2 (the number of permutations of a set with $n$ elements is $n!$).

\hrulefill\vspace{2mm}

\begin{enumerate}[label=\big(\alph*\big)]
    \item We're going to use as the induction hypothesis that the number of strings of length $(n-1)$ composed of $k$ given elements is $k^{n-1}$.
    With this in mind, let's formulate what would be the formula for strings of length $n$.
    \begin{align*}
        S &= k_1 \cdot k_2 \cdot \ldots \cdot k_{n-1} \cdot k_n \\
        S &= [k_1 \cdot k_2 \cdot \ldots \cdot k_{n-1}] \cdot k_n \\
        S &= k^{n-1} \cdot k_n \\
        S &= k^n
    \end{align*}
    \item This is quite similar to what we did before. Taking as the induction hypothesis that the number of permutations of $(n-1)$ objects is $(n-1)!$, we can develop our formula for $n$ permutations like this
    \begin{align*}
        P &= (n-1)! \cdot n \\
        P &= n!
    \end{align*}
    This is true because the $n$th element can be inserted in any of $n$ positions.
\end{enumerate}

\subsection{Exercise}

Use induction on $n$ to prove the ``handshake theorem'' (the number of handshakes
between $n$ people in $n(n-1)/2$).

\hrulefill\vspace{2mm}

Using as the induction hypothesis that the number of handshakes between $(n-1)$ people is $(n-1)(n-2)/2$ we get that the $n$th person can handshake once each one of the $(n-1)$ persons.
With this in mind, let's expand the formula.

\begin{align*}
    \frac{(n-1)(n-2)}{2} + (n-1) &= \frac{(n-1)(n-2)+2(n-1)}{2} \\
    \frac{(n-1)(n-2)}{2} + (n-1) &= \frac{(n-1)[(n-2)+2]}{2} \\
    \frac{(n-1)(n-2)}{2} + (n-1) &= \frac{n(n-1)}{2}
\end{align*}

\subsection{Exercise}

Read carefully the following induction proof: \\

Assertion: $n(n+1)$ is an odd number for every $n$. \\

Proof: Suppose that this is true for $n-1$ in place of $n$; we prove it for $n$, using the
induction hypothesis. We have
\begin{align*}
    n(n+1) = (n-1)n + 2n.
\end{align*}
Now here $(n-1)n$ is odd by the induction hypothesis, and $2n$ is even. Hence $n(n+1)$ is the sum of an odd number and an even number, which is odd. \\

The assertion that we proved is obviously wrong for $n = 10$: $10 \cdot 11 = 110$ is even. What is wrong with the proof?

\hrulefill\vspace{2mm}

The problem here is that there's no base case for the induction. As we see, if $n=1$ then the hypothesis ($n-1=0$) becomes $0(0+1)=0$, which it's even. As the induction hypothesis is incorrect, the whole proof becomes invalid.

\subsection{Exercise}

Read carefully the following induction proof: \\

Assertion: If we have $n$ lines in the plane, no two of which are parallel, then they all
go through one point. \\

Proof: The assertion is true for one line (and also for 2, since we have assumed that
no two lines are parallel). Suppose that it is true for any set of $n-1$ lines. We are
going to prove that it is also true for $n$ lines, using this induction hypothesis. \\
So consider a set of $S = \{a, b, c, d, \dots\}$ of $n$ lines in the plane, no two of which are
parallel. Delete the line $c$, then we are left with a set $S'$ of $n-1$ lines, and obviously no
two of these are parallel. So we can apply the induction hypothesis and conclude that
there is a point $P$ such that all the lines in $S'$ go through $P$. In particular, $a$ and $b$ go through $P$, and so $P$ must be the point of intersection of $a$ and $b$. \\
Now put $c$ back and delete $d$, to get the set $S''$ of $n-1$ lines. Just as above, we can use
the induction hypothesis to conclude that these lines go through the same point $P'$; but
just like above, $P'$ must be the point of intersection of $a$ and $b$. Thus $P' = P$. But then
we see that $c$ goes through $P$. The other lines also go through $P$ (by the choice of $P$), and so all the $n$ lines go through $P$. \\

But the assertion we proved is clearly wrong; where is the error?

\hrulefill\vspace{2mm}

The proof is wrong because it's using the induction hypothesis in a situation where it doesn't apply.
Basically, the point of intersection $P$ and $P'$ may not be the same, as the hypothesis never tells us that they must be the same.
So then, the induction step fails as the hypothesis isn't useful enough to prove the general case.

\subsection{Exercise}

Describe a proof of Theorem 3.1 using induction on the number of lines.

\hrulefill\vspace{2mm}

Theorem 3.1 says like this: a set of $n$ lines in general position in the plane divides the plane into $1 + n(n+1)/2$ regions.
To solve this, we're going to use as the induction hypothesis that it is true for $(n-1)$, so
\begin{align*}
    1 + [1 + 2 + 3 + \cdots + (n-1)] + n &= 1 + \frac{(n-1)n}{2} + n \\
    1 + [1 + 2 + 3 + \cdots + (n-1)] + n &= 1 + \frac{(n-1)n + 2n}{2} \\
    1 + [1 + 2 + 3 + \cdots + (n-1)] + n &= 1 + \frac{n^2 - n + 2n}{2} \\
    1 + [1 + 2 + 3 + \cdots + (n-1)] + n &= 1 + \frac{n^2 + n}{2} \\
    1 + [1 + 2 + 3 + \cdots + (n-1)] + n &= 1 + \frac{n(n+1)}{2}
\end{align*}
With this, we proved Theorem 3.1 using induction.

\section{Counting subsets}

\subsection{Exercise}

Illustrate this argument by a tree.

\hrulefill\vspace{2mm}

The argument says something like this: if in a competition there's 100 athletes and only the order of the first 10 is recorded, then there's $100 \cdot 99 \cdot \ldots \cdot 91$ different outcomes for the competition.
The tree to illustrate this argument is really big, so i'm only going to do it partially. \\
% Really, with this kind of exercises I don't really understand what they want me to do.
\\
\begin{forest}
    [$100$ [$99$ [$\cdots$]] [$98$ [$\cdots$]] [$97$ [$\cdots$]] [$96$ [$\cdots$]] [$95$ [$\cdots$]] [$94$ [$\cdots$]] [$93$ [$\cdots$]] [$92$ [$\cdots$]] [$91$ [$\cdots$]] [$90$ [$\cdots$]] [$89$ [$\cdots$]] [$88$ [$\cdots$]] [$\cdots$ [$\cdots$]]]
\end{forest} \\
\begin{forest}
    [$99$ [$99$ [$\cdots$]] [$98$ [$\cdots$]] [$97$ [$\cdots$]] [$96$ [$\cdots$]] [$95$ [$\cdots$]] [$94$ [$\cdots$]] [$93$ [$\cdots$]] [$92$ [$\cdots$]] [$91$ [$\cdots$]] [$90$ [$\cdots$]] [$89$ [$\cdots$]] [$88$ [$\cdots$]] [$\cdots$ [$\cdots$]]]
\end{forest} \\
\begin{forest}
    [$98$ [$99$ [$\cdots$]] [$98$ [$\cdots$]] [$97$ [$\cdots$]] [$96$ [$\cdots$]] [$95$ [$\cdots$]] [$94$ [$\cdots$]] [$93$ [$\cdots$]] [$92$ [$\cdots$]] [$91$ [$\cdots$]] [$90$ [$\cdots$]] [$89$ [$\cdots$]] [$88$ [$\cdots$]] [$\cdots$ [$\cdots$]]]
\end{forest} \\
\begin{forest}
    [$\cdots$ [$99$ [$\cdots$]] [$98$ [$\cdots$]] [$97$ [$\cdots$]] [$96$ [$\cdots$]] [$95$ [$\cdots$]] [$94$ [$\cdots$]] [$93$ [$\cdots$]] [$92$ [$\cdots$]] [$91$ [$\cdots$]] [$90$ [$\cdots$]] [$89$ [$\cdots$]] [$88$ [$\cdots$]] [$\cdots$ [$\cdots$]]]
\end{forest}

\subsection{Exercise}

Suppose that we record the order of all 100 athletes.

\begin{enumerate}[label=\big(\alph*\big)]
    \item How many different outcomes can we have then?
    \item How many of these give the same for the first 10 places?
    \item Show that the result above for the number of possible outcomes for the first 10 places can be also obtained using (a) and (b).
\end{enumerate}

\hrulefill\vspace{2mm}

\begin{enumerate}[label=\big(\alph*\big)]
    \item The number of outcomes is a permutation, like the ones we saw in chapter 2.5.
    \begin{align*}
        100! = 100 \cdot 99 \cdot \ldots \cdot 1
    \end{align*}
    \item My guess is that, because we can't change the first 10, we are only ordering the last 90. So the result would be
    \begin{align*}
        90! = 90 \cdot 89 \cdot \ldots \cdot 1
    \end{align*}
    \item In effect, the number of possible outcomes for the first 10 places is
    \begin{align*}
        \frac{100!}{90!} &= \frac{100 \cdot 99 \cdot \ldots \cdot 1}{90 \cdot 89 \cdot \ldots \cdot 1} \\
        \frac{100!}{90!} &= 100 \cdot 99 \cdot \ldots \cdot 91
    \end{align*}
\end{enumerate}

\subsection{Exercise}

If you generalize the solution of exercise 4.1, you get the answer in the form
\begin{align*}
    \frac{n!}{(n-k)!}
\end{align*}
Check that this is the same number as given in theorem 4.1.

\hrulefill\vspace{2mm}

I've already used that formula in the last exercise, so I know it's correct. But let's prove it using the theorem.

\begin{align*}
    \frac{n!}{(n-k)!} &= \frac{n \cdot (n-1) \cdot \ldots \cdot 1}{(n-k) \cdot [(n-k)-1] \cdot \ldots \cdot 1} \\
    \frac{n!}{(n-k)!} &= \frac{n \cdot (n-1) \cdot \ldots \cdot [(n-k) + 1] \cdot (n-k) \cdot [(n-k) - 1] \cdot \ldots \cdot 1}{(n-k) \cdot [(n-k)-1] \cdot \ldots \cdot 1} \\
    \frac{n!}{(n-k)!} &= n \cdot (n-1) \cdot \ldots \cdot [(n-k) + 1] \cdot \frac{(n-k) \cdot [(n-k) - 1] \cdot \ldots \cdot 1}{(n-k) \cdot [(n-k)-1] \cdot \ldots \cdot 1} \\
    \frac{n!}{(n-k)!} &= n \cdot (n-1) \cdot \ldots \cdot [(n-k) + 1]
\end{align*}

\subsection{Exercise}

Explain the similarity and the difference between the counting questions answered
by theorem 4.1 and theorem 2.2.

\hrulefill\vspace{2mm}

The difference is that theorem 2.2 talks about strings of length $n$ composed of $k$ given elements. That is to say, it opens the possibility to repeating elements in the string.
Meanwhile, theorem 4.1 is about ordered subsets of a set. That means that elements shouldn't repeat, as they are specific elements of the subset.
They're quite similar in the sense that they produce a kind of sequence of elements, but those sequences are different in nature.

\subsection{Exercise}

Which problems discussed during the party were special cases of theorem 4.2?

\hrulefill\vspace{2mm}

The handshake problem can be seen like this
\begin{align*}
    \binom{7}{2} = \frac{7!}{2!(7-2)!} = 21
\end{align*}
The dancing partner problem can also be seen like this
\begin{align*}
    3 \cdot \binom{4}{3} = 4 \cdot \frac{4!}{3!(4-3)!} = 12
\end{align*}
The lottery problem can be seen like this
\begin{align*}
    \binom{90}{5} = \frac{90!}{5!(90-5)!} = 43949268
\end{align*}
The bridge hand problem can also be seen like this
\begin{align*}
    \binom{52}{13} = \frac{52!}{13!(52-13)!} = 635013559600
\end{align*}
And, although a bit different from how they solved it, the chess board problem can also be seen like this
\begin{align*}
    \binom{6}{2} = \frac{6!}{2!(6-2)!} = 15
\end{align*}

\subsection{Exercise}

Tabulate the values of $\binom{n}{k}$ for $n,k \leqslant 5$.

\hrulefill\vspace{2mm}

I've omitted the cases where $k > n$, because it's 0 and it repeats quite often if I tabulate all the possibilities.

\begin{center}
    \begin{tabular}{||c c c||} 
    \hline
        $n$ & $k$ & $\binom{n}{k}$ \\ [0.5ex] 
        \hline\hline
        0 & 0 & 1 \\ 
        \hline
        1 & 0 & 1 \\
        \hline
        1 & 1 & 1 \\
        \hline
        2 & 0 & 1 \\
        \hline
        2 & 1 & 2 \\
        \hline
        2 & 2 & 1 \\
        \hline
        3 & 0 & 1 \\
        \hline
        3 & 1 & 3 \\
        \hline
        3 & 2 & 3 \\
        \hline
        3 & 3 & 1 \\
        \hline
        4 & 0 & 1 \\
        \hline
        4 & 1 & 4 \\
        \hline
        4 & 2 & 6 \\
        \hline
        4 & 3 & 4 \\
        \hline
        4 & 4 & 1 \\
        \hline
        5 & 0 & 1 \\
        \hline
        5 & 1 & 5 \\
        \hline
        5 & 2 & 10 \\
        \hline
        5 & 3 & 10 \\
        \hline
        5 & 4 & 5 \\
        \hline
        5 & 5 & 1 \\
        \hline
    \end{tabular}
\end{center}

\end{document}
